%\lecture{2}{Tue 13 Jul 2021 14:09}{Chevalley-Warning Theorem and Sumsets}
\chapter{Simple Combinatorial Proofs}
\section{Sumsets}\todo{prove 1-2 more combinatorial statements with Nullstellensatz} Now, with our the Nullstellensatz stated and proven, let us examine a few simple results concerning sumsets, those being the pointwise sum of all coordinates
\begin{theorem}[Cauchy-Davenport Theorem \cite{alon_1999}] Given a
	prime $p$ and nonempty $A,B \subseteq Z_{p}$, then $\left| A + B
	\right| \ge \min \left\{ p, \left| A \right|  + \left| B \right|  -1
	\right\} $.  \end{theorem} \begin{proof}[Proof] Suppose $\left| A
	\right| + \left| B \right| > p$, we must have that for any element $x
	\in Z_p$, $\left( A \right) \cap \left( x \setminus B \right) \neq \O$
	(As there are only $p$ possible elements which could be in each set).
	Hence, $A + B = Z_p$. Thus, let us assume $\left| A \right|  + \left| B
	\right| \le p$ and suppose indirectly that $\left| A+B \right| \le
	\left| A \right| + \left| B \right| -2$. Let $A+B \subseteq C \subseteq
	Z_p$ such that $\left| C \right| = \left| A \right|  + \left| B \right|
	-2$. Next, define $f\left( x,y \right)  = \prod_{c \in C}^{} \left( x +
	y - c \right) $ and note that we must have $f\left( a,b \right)  = 0$
	for all $\left( a,b \right)  \in A \times B$ as $A,B \subseteq C$. Now, note
	that $\deg \left( f \right) = \left| C \right| = \left| A \right|  +
	\left| B \right|  -2$, and hence $[x^{\left| A \right| -1}y ^{\left| B
	\right| -1}]f\left( x,y \right) = \binom{\left| A \right| + \left| B
\right|  -2}{\left| A \right| -1} \neq 0$ as $\left| A \right| -1 < \left| A
\right|  + \left| B \right| -2 < p$. Hence, by the Combinatorial
Nullstellensatz (Theorem 1.2), we must have a pair $\left( a,b \right)  \in A
\times B$ such that $f\left( a,b \right) \neq 0$ $\lightning$. Thus, the theorem must be true.
\end{proof} With this basic theorem about
sumsets proven, we now take a look at restricted sumsets, those being sumsets
where a sum is excluded if it satisfies a certain property, normally being the
root of a particular polynomial.  \begin{notation}[Restricted Sumset] For a
	polynomial $h \left( x_0, x_1, \ldots, x_{k} \right) $ and subsets
	$A_0, A_1, \ldots, A_{k} \subseteq Z_p$ define \[ \oplus _h
	\sum_{i=0}^{k} A_{i} = \left\{ a_0 + a_1 + \ldots + a_{k} : a_{i} \in
A_{i}, h\left( a_0, a_1, \ldots, a_{k} \right) \neq 0 \right\} \] to be the
restricted sumset over the $A_{i}$s with respect to $h$.  \end{notation}
\begin{theorem}[General Restricted Sumset Theorem\cite{alon_1999}] For a prime $p$ and a
	polynomial $h\left( x_0, .., x_{k} \right)$ over $Z_p$ and nonempty
	$A_0, A_1, \ldots, A_{k} \subseteq Z_p$, define  $c_{i} = \left| A_{i}
	\right| -1$ and $m = \sum_{i=0}^{k} c_{i} - \deg \left( h \right)$.\\
	If $[x_0^{c_0}\ldots x_{k}^{c_{k}}]((\sum_{i=0}^{k} x_{i})^{m} \cdot
	h\left( \textbf{x} \right) ) \neq 0$, then \[ \left|  \oplus _h
		\sum_{i=0}^{k}  A_{i} \right| \ge m+1 \ \text{ (consequently m
		< p)} .\] \end{theorem} \begin{proof}[Proof] Suppose indirectly
		that the inequality does not hold (and hence $m \ge p$), then
		we may define $E \subseteq Z_p$ such that $E$ is a multi-set
		containing $m$ elements and $\oplus_h \sum_{i=0}^{k} A_{i}
		\subseteq E$. Let $Q\left( \textbf{x} \right) = h\left(
			\textbf{x} \right) \prod_{e \in E}^{}  \left(
			\sum_{i=0}^{k} x_{i} - e \right)  $.  By our
			construction we must have that $Q\left( \textbf{x}
			\right)  = 0$ for all $\textbf{x} \in \prod_{i=0}^{k}
			A_{i}$ as either $h\left( \textbf{x} \right)  = 0$ or
			$\sum_{i=0}^{k} x_{i} \in \oplus_h \sum_{i=0}^{k}
			A_{i} \subseteq E$. Furthermore $\deg \left( Q \right)
			= m + \deg \left( h \right) = \sum_{i=0}^{k} c_{i}$ by
			construction. From this we see $m \ge \sum_{i=0}^{k}
			c_{i}$ and hence $[x_0^{c_0}\ldots c_{k}^{c_{k}}] Q
			\neq 0$ (as it is binomial in nature).\\ Therefore,
			applying Combinatorial Nullstellenstaz (Theorem 1.2)
			yields an $ \textbf{a}\in A$ such at $Q\left(
			\textbf{a} \right) \neq 0 $ $\lightning$. Thus $m < p$
			and $\left| \oplus _ h \sum_{i=0}^{k}  A_{i}  \right|
			\ge m+1$.  \end{proof} With this powerful result proven
			let us now take specific functions for $h$ and prove
			superior lower bounds where possible. First, we examine
			the function $h\left( a_0, \ldots, a_{k} \right)  =
			\prod_{0\le i < j \le n}^{} \left( a_{i} - a_{j}
			\right) $: \begin{theorem}[Restricted Sumset Theorem\cite{alon_1999}]
				For a prime $p$, nonempty $A_0, A_1, \ldots,
				A_{k} \subseteq Z_p$ with  $\left| A_{i}
				\right| \neq \left| A_{j} \right| $ for any $i
				\neq j$ and for the $h$ defined above. If
				$\sum_{i=0}^{k} \left| A_{i} \right| \le p +
				\binom{k+2}{2} -1$, then \[ \left| \oplus_h
				\sum_{i=0}^{k} A_{i} \right| \ge \sum_{i=0}^{k}
			\left| A_{i} \right| - \binom{k+2}{2} + 1 .\]
		\end{theorem} A special case of this theorem for only two sets
		is the following: \begin{theorem}[Erdős-Heilbronn Conjecture\cite{alon_1999}]
			For a prime $p$ and nonempty $A,B \subseteq Z_{p}$,
			then $\left| A \oplus_{h} B \right| \ge \min \left\{ p,
			\left| A \right|  + \left| B \right|  - \delta \right\}
			$ where $\delta = 3$ for  the case $A = B$ and $\delta
			= 2$ in all other cases.  \end{theorem} In order to
			prove these theorems, let us first state and prove a
			lemma concerning the coefficient of a particular
			polynomial: \begin{lemma}[] Let $0 \le c_0, \ldots,
				c_{k} \in \Z$ and define $m = \sum_{i=0}^{k}
				c_{i} - \binom{k+1}{2}$ (it is trivial that $m$
				is nonnegative). Then,\[[x_0^{c_0}\ldots
				x_{k}^{c_{k}}] \left( \left( \sum_{i=0}^{k}
			x_{i} \right) ^{m} \prod_{k \ge i > j \ge 0}^{} \left(
x_{i} - x_{j} \right)  \right) = \frac{m!}{c_0! c_1! \ldots c_{k}!} \prod_{k
\ge i > j \ge 0}^{} \left( c_{i} - c_{j} \right).\] \end{lemma} \todo{Provide
proof of this lemma based on proof of Ballot problem} With this out of the way,
we now prove proposition 2.4: \begin{proof}[Proof of Restricted Sumset Theorem ]
	For this proof we will take the aforementioned \[ h\left( \textbf{x}
		\right) = \prod_{0\le i < j \le k}^{} \left( x_{i} - x_{j}
	\right) .\] Now, let us define $c_{i} = \left| A_{i} \right|  -1$ and
	$m = \sum_{i=0}^{k} c_{i} - \binom{k+1}{2}$. Rearranging the
	assumptions of this theorem yields $\sum_{i=0}^{k} \left| A_{i} \right|
	- \binom{k+2}{2} + 1 \le p$ and, applying the trivial combinatorial
	identity $\binom{k+2}{2} =  \binom{k+1}{2} +  (k+1)$ yields \\
	$\sum_{i=0}^{k} c_{i} - \binom{k+1}{2} + 1 = m + 1 \le p$ (hence $m <
	p$). Then  \[ [x_0^{c_0}\ldots x_{k}^{c_{k}}]\left( \left(
	\sum_{i=0}^{k} x_{i} \right) h \right) =  \frac{m!}{c_0! \ldots
	c_{k}!}\prod_{0 \le i < j \le k}^{} \left( c_{i} - c_{j} \right) .\] We
	know this product to be nonzero modulo  $p$ as  $c_{i} \neq c_{j}$  for
	$i \neq j$ by construction and $m < p$. Finally, as the coefficient is
	nonzero and as $\deg \left( h \right) =  \binom{k+2}{2}$ (as there are
	$k+2$ possible $x_{i}$'s and each term of the product will contain two
	distinct $x_{i}$'s so there are $\binom{k+2}{2}$ terms each of degree
	$1$), we have $m = \sum_{i=0}^{k} c_{i} - \deg \left( h \right)$
	\todo{Ask grynkiewicz about this as Alons paper says $+ \deg \left( h
	\right)$ instead of minus}. Hence, applying theorem 2.3 yields  $\left|
	\oplus _h \sum_{i= 0}^{k}  A_i \right| \ge m + 1 = \sum_{i=0}^{k}
	\left| A_{i} \right| - \binom{k+2}{2}+1$ by construction.  \end{proof}
We conclude this section with the proof of the Erdős-Heilbronn Conjecture:
\begin{proof}[Proof of Erdős-Heilbronn Conjecture]
	\todo{Add proof}
\end{proof}
\newpage
\section{Graphs, Cubes, and Colorings}
Now, we present a few combinatorial results of graph theory using the Combinatorial Nullstellensatz. For these proofs assume all graphs are loopless,
\begin{theorem}[\cite{alon_1999}]
Let \(p\) be an odd prime and \(G\) ,a graph with \(\frac{\sum_{i \in V\left( G \right) }^{} d\left( i \right)}{e\left( G \right) } = d\left( G \right) > 2p-2\) and \(\Delta \le 2p-1\), we find a \(p\)-regular subgraph.\end{theorem}
\begin{proof}
	Let \(B = \left( b_{i, e} \right)_{i \in V\left( G \right) , e \in E\left( G \right) } \) to be the incidence matrix of \(G\), that being,  \(b_{i, j} = \left \{
		\begin{array}{11}
			1, & \quad i \in e  \\
			0, & \quad i \not\in e
		\end{array}
		\right\).
		For each \(e \in E\left( G \right) \), let \(x_{e} \in \{0, 1\}  \) be an associated variable and define \begin{align*}
			F: \{0, 1\} ^{e\left( G \right) } &\longrightarrow \GF\left( p \right)  \\
			(x_{e})_{e \in E\left( G \right) } 	&\longmapsto F((x_{e})_{e \in E\left( G \right)} ) = \prod_{v \in V}^{}\left[ 1 - \left( \sum_{e \in E}^{} b_{v, e}x_{e} \right)^{p-1}  \right] - \prod_{e \in E}^{} \left( 1- x_{e} \right)
	.\end{align*}		Then, recall that \(\frac{d\left( G \right) \cdot v\left( G \right) }{2} = e\left( G \right) \). Hence, we have \(e\left( G \right)  > \frac{2p-2}{2}v\left( G \right)  = \left( p-1 \right) v\left( G \right)  \). And, as the highest order term of the first product has degree \(\left( p-1 \right)\) as each \(x_{e}\) within the sum has degree \(p-1\) and the product over \(v\left( G \right) \) terms of order \(p-1\) is \(v\left( G \right) \left( p-1 \right) \), and the highest order term of the second product is \(\left( -1 \right) ^{e\left( G \right) -1} \prod_{e \in E\left( G \right) }^{} x_{e} }\) of order \(e\left( G \right) \), hence \(\deg \left( F \right) = e\left( G \right) \). Then, as the highest order term is \(\prod_{e \in E\left( G \right) }^{}  x_{e}^{1}\) with each \(x_{e} \in \{0, 1\} \), hence as \(\left| \{ 0, 1 \} \right|  - 1= \deg \left( x_{e} \right)  \), we see that applying combinatorial nullstellensatz yields an \(\left( x_{e} \right) _{e \in E\left( G \right) = \textbf{x} }\) such that \(F\left( \textbf{x} \right) \neq 0\) and as \(F\left( \left( 0 \right) _{e \in E\left( G \right) } \right) = 0 \), we see \(\textbf{x} \neq \left( 0 \right) _{e \in E\left( G \right) }\). Hence, \(\prod_{\textbf{x}}^{}  \left( 1 - x_{e}\right) = 0 \), so we see the first product is nonzero. Then, as \(a^{p -1} \equiv 1 \left( \mod p \right) \)  for all \(b \neq 0\), we see each \(\sum_{e \in E\left( G \right) }^{} b_{v, e}x_{e} = 0\), else the first product would be zero.\\
	Now, let \(H\) be the subgraph induced by \(E\left( H \right) = \{e\in E\left( G \right) : x _{e} = 1\} \) and note that as all terms of the sum \(\sum_{e \in E}^{} b_{v, e} x_{e}\) are either \(0\) or \(1\) and as there are precisely \(2\) \(v \in V\left( G \right) \) such that \(b_{v, e} = 1\) for each \(e \in E\left( G \right) \), we see \(p \mid \left| \{e \in E\left( G \right) : x_{e} = 1\} \right\), as \(p\) is an odd prime. Then, as \(x_{e} = 1\) for all \(e \in E\left( H \right) \), we see \(\sum_{e \in E\left( H \right) }^{} a_{v, e}x _{e} = \sum_{e \in E\left( H \right) }^{} a_{v, e} =  d_{H}\left( v \right) \) and as \(p \mid e\left( H \right) = \left|\{e \in E\left( G \right) : x_{e} = 1\}  \right| \), we see \(p \mid d_{H}\left( v \right) \) for each \(v \in V\left( H \right) \). Furthermore, as \(d\left( v \right)  < 2p\) for all \(v \in H\), we see \(d_{H}\left( v \right)  = p\) for all \(v \in V\left( H \right) \). Hence, \(H\) is \(p\)-regular.

\end{proof}

\begin{theorem}[\cite{alon_1999}]
	Let \(p\) be a prime and let \(G = \left( V, E \right) \) be a graph with \(v\left( G \right)  > d\left( p-1 \right) \). Then, there is a nonempty \(U \subseteq V\left( G \right) \) such that the number of cliques on \(d\) vertices of \(G\) intersecting \(U\) is a multiple of \(p\)b

\end{theorem}
