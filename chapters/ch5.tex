%\lecture{2}{Tue 13 Jul 2021 14:09}{Chevalley-Warning Theorem and Sumsets}
\chapter{Chevalley-Warning Theorem and Kemnitz Conjecture} \todo{Move Chevalley-Warning to
its own section and proved 1-2 more statements and proofs on sumsets} Our first
major theorem of this section concerns the theorem of Chevalley and Warning
which declares the conditions under which a certain nontrivial solution to a
polynomial in a finite field of characteristic $p$ can exist:
\begin{theorem}[Chevalley-Warning Theorem] Let $F$ be a finite field of
	characteristic $p$ and let $f_1, f_2, \ldots, f_k \in F[x_1, x_2,
	\ldots, x_{n}]$ be polynomials and $N$ to be the number of points
	$\textbf{x}\in F^{n}$ such that $f_1\left( \textbf{x} \right) = \ldots
	= f_k \left( \textbf{x} \right)  = 0$. If $ \sum_{i= 1}^{k} \deg \left(
	f_{i} \right)< n$, then $N \equiv 0 (\text{mod } p) $.  \end{theorem}
	In order to provide a proof of this statement, let us first state and
	prove the following lemma: \begin{lemma}[Lemma] Let $F$ be a finite
		field and $k_1, k_2, \ldots, k_{n} \ge 0$ such that $\min _{1
		\le i \le n} k_{i} \le \left| F \right|  - 2$. Then,
		$\sum_{x_1, x_2, \ldots, x_{n} \in F}^{}
		x_1^{k_1}x_2^{k_2}\ldots x_{n}^{k_{n}} = 0.$ (Note: if a
		$0^{0}$ occurs in the expressions it will be treated as a $1$).
		\end{lemma} \begin{proof}[Proof] Assume without loss of
		generality that $k_1 < \left| F  \right| -1$. Then, by factoring out a $x_1^{k_1}$ from each term of the sum and grouping all such $x_1$'s, we have
		$\sum_{x_1, x_2, \ldots, x_{n} \in F}^{}
		x_1^{k_1}x_2^{k_2}\ldots x_{n}^{k_{n}} = \left( \sum_{x_1 \in
		F}^{} x_1^{k_1} \right) \sum_{x_2, \ldots, x_{n} \in F}^{} x_2^{k_2}\ldots x_{n} ^{ k_{n}}$, hence we must
		only show that $\sum_{x_1 \in F}^{} x_1^{k_1}= 0$. Suppose $k_1
		= 0$, then $\sum_{x_1 \in F}^{}x_1^{k_1} = \left| F \right| $ and,
		since $p$ divides  $\left|  F \right|$ we see the case $k_1 =
		0$ is trivially true. Now, let $\omega \in F^{\times}$ be a
		generator of $F^{\times}$. Then, we have that \[ \sum_{x_1 \in
			F}^{}x_1 ^{ k_1} = \sum_{x_1 \in F^{\times}}^{}x_1 ^{
		k_1} = \sum_{x_1 \in F^{\times}}^{}\left( \omega x \right) ^{
	k_1} = \omega ^{ k_1} \sum_{x_1 \in F^{ \times}}^{} x_1 ^{ k_1}= \omega
^{ k_1}\sum_{x_1 \in F}^{}x_1 ^{ k_1} .\] Taking the difference of the first and last terms of the above equality yields $\left(
	\omega ^{ k_1} - 1 \right) \left( \sum_{x_1 \in F}^{} x_1 ^{
	k_1}\right) = 0$, so we must have either the sum is $0$ or $\omega
	^{k_1} -1 = 0$. However, as $\omega$ is a generator of the cyclic group
	$F^{ \times}$, we may only have that $\omega ^{ k_1} = 1$ if $k_1
\equiv -1 \left( \text{mod } \left| F \right|  \right) $.  But, as $0 < k_1 <
\left| F \right| -1$ this case cannot occur, hence we see $\sum_{x_1 \in F}^{}
x_1 ^{ k_1} = 0$, so the lemma is proven.  \end{proof} \begin{proof}[Proof of
Chevalley-Warning Theorem] Recalling that $x^{\left|  F \right| } = x$ (and
thus $x^{\left| F \right| -1} = 1$ for nonzero $x$). Then, define  \[ M =
	\sum_{\textbf{x} \in F^{n}}^{} \prod_{i= 1}^{k} \left( 1 - f_{i}\left(
\textbf{x} \right) ^{\left| F \right| -1} \right) .\] We see, by the earlier
proposition, that a term of the sum will be $1$ if and only if $\textbf{x}$ is
a solution to the system $f_1, f_2, \ldots, f_{k}$, else it will be $0$.
Furthermore, it is clear by the construction that $M$ will be exactly equal to
the number of solutions to our system $f_1, \ldots, f_{k}$, and hence it
precisely $N$.\\ Now, let us define the product from our construction to be a
polynomial  $g$, that is $g\left( \textbf{x} \right) = \prod_{i= 1}^{k} \left(
	1-f_{i}\left( \textbf{x} \right) ^{\left| F \right| -1} \right)$. Then,
	repeatedly applying the substitution $ x_{j}^{\left| F \right| } \to
	x_{j}$ to $g$ yields a polynomial $\overline{g} = g$ for all
	$\textbf{x}\in F^{n}$. Furthermore, $ \deg _{ x_j} \left(  \overline{g}
	\right) \le \left| F \right| -1$ for $1 \le j \le n$(This is clear as ,
	if it were not we would be able to apply the substitution once again).
	Then, substituting $\overline{g}$ in place of $g$ yields \[ M =
	\sum_{\textbf{x } \in F^{n}}^{} \overline{g}\left( \textbf{x} \right)
.\] Then, applying our lemma, we see that all monomials with degree $\left| F
\right|  -2$ or less will equal $0$ and hence the only possible nonzero terms
of $\overline{g}$ are those of the form $ \prod_{i= 1}^{n}  x_{i}^{\left| F
\right| -1}$. Expanding the product, we see that such a monomial would be of
degree $n\left(\left| F \right| - 1 \right)$, however as $\deg f_{i} ^{\left| F
\right| -1} = \left( \left| F \right| -1 \right) \deg \left( f_{i} \right)$, we
see that $\deg \left( g \right) \le \left( \left| F \right| -1 \right) \sum_{i=
1}^{k} f_{i} < n\left( \left| F \right| -1 \right)$ by construction.
Consequently, any such monomial of $\overline{g}$ (and hence $g$) will have a
zero coefficient, and thus $M = N \equiv 0 \left( \text{mod } p \right) $.
\end{proof} This theorem allows us to prove certain constraints (and in the
case of prime  $p$, the exact value) on the size of the davenport constant for
a certain group. This result is too advanced for now, but we will return to it
later. Next, let us examine theorems concerning sumsets such as the
Cauchy-Davenport Theorem and the Erdős-Heilbronn conjecture on the lower bound
of the size of sumsets.
