%\lecture{2}{Tue 13 Jul 2021 14:09}{Chevalley-Warning Theorem and Sumsets}
\chapter{Chevalley-Warning Theorem and Sumsets} \todo{Move Chevalley-Warning to
its own section and proved 1-2 more statements and proofs on sumsets} Our first
major theorem of this section concerns the theorem of Chevalley and Warning
which declares the conditions under which a certain nontrivial solution to a
polynomial in a finite field of characteristic $p$ can exist:
\begin{theorem}[Chevalley-Warning Theorem] Let $F$ be a finite field of
	characteristic $p$ and let $f_1, f_2, \ldots, f_k \in F[x_1, x_2,
	\ldots, x_{n}]$ be polynomials and $N$ to be the number of points
	$\textbf{x}\in F^{n}$ such that $f_1\left( \textbf{x} \right) = \ldots
	= f_k \left( \textbf{x} \right)  = 0$. If $ \sum_{i= 1}^{k} \deg \left(
	f_{i} \right)< n$, then $N \equiv 0 (\text{mod } p) $.  \end{theorem}
	In order to provide a proof of this statement, let us first state and
	prove the following lemma: \begin{lemma}[Lemma] Let $F$ be a finite
		field and $k_1, k_2, \ldots, k_{n} \ge 0$ such that $\min _{1
		\le i \le n} k_{i} \le \left| F \right|  - 2$. Then,
		$\sum_{x_1, x_2, \ldots, x_{n} \in F}^{}
		x_1^{k_1}x_2^{k_2}\ldots x_{n}^{k_{n}} = 0.$ (Note: if a
		$0^{0}$ occurs in the expressions it will be treated as a $1$).
		\end{lemma} \begin{proof}[Proof] Assume without loss of
		generality that $k_1 < \left| F  \right| -1$. Then, by factoring out a $x_1^{k_1}$ from each term of the sum and grouping all such $x_1$'s, we have
		$\sum_{x_1, x_2, \ldots, x_{n} \in F}^{}
		x_1^{k_1}x_2^{k_2}\ldots x_{n}^{k_{n}} = \left( \sum_{x_1 \in
		F}^{} x_1^{k_1} \right) \sum_{x_2, \ldots, x_{n} \in F}^{} x_2^{k_2}\ldots x_{n} ^{ k_{n}}$, hence we must
		only show that $\sum_{x_1 \in F}^{} x_1^{k_1}= 0$. Suppose $k_1
		= 0$, then $\sum_{x_1 \in F}^{}x_1^{k_1} = \left| F \right| $ and,
		since $p$ divides  $\left|  F \right|$ we see the case $k_1 =
		0$ is trivially true. Now, let $\omega \in F^{\times}$ be a
		generator of $F^{\times}$. Then, we have that \[ \sum_{x_1 \in
			F}^{}x_1 ^{ k_1} = \sum_{x_1 \in F^{\times}}^{}x_1 ^{
		k_1} = \sum_{x_1 \in F^{\times}}^{}\left( \omega x \right) ^{
	k_1} = \omega ^{ k_1} \sum_{x_1 \in F^{ \times}}^{} x_1 ^{ k_1}= \omega
^{ k_1}\sum_{x_1 \in F}^{}x_1 ^{ k_1} .\] Taking the difference of the first and last terms of the above equality yields $\left(
	\omega ^{ k_1} - 1 \right) \left( \sum_{x_1 \in F}^{} x_1 ^{
	k_1}\right) = 0$, so we must have either the sum is $0$ or $\omega
	^{k_1} -1 = 0$. However, as $\omega$ is a generator of the cyclic group
	$F^{ \times}$, we may only have that $\omega ^{ k_1} = 1$ if $k_1
\equiv -1 \left( \text{mod } \left| F \right|  \right) $.  But, as $0 < k_1 <
\left| F \right| -1$ this case cannot occur, hence we see $\sum_{x_1 \in F}^{}
x_1 ^{ k_1} = 0$, so the lemma is proven.  \end{proof} \begin{proof}[Proof of
Chevalley-Warning Theorem] Recalling that $x^{\left|  F \right| } = x$ (and
thus $x^{\left| F \right| -1} = 1$ for nonzero $x$). Then, define  \[ M =
	\sum_{\textbf{x} \in F^{n}}^{} \prod_{i= 1}^{k} \left( 1 - f_{i}\left(
\textbf{x} \right) ^{\left| F \right| -1} \right) .\] We see, by the earlier
proposition, that a term of the sum will be $1$ if and only if $\textbf{x}$ is
a solution to the system $f_1, f_2, \ldots, f_{k}$, else it will be $0$.
Furthermore, it is clear by the construction that $M$ will be exactly equal to
the number of solutions to our system $f_1, \ldots, f_{k}$, and hence it
precisely $N$.\\ Now, let us define the product from our construction to be a
polynomial  $g$, that is $g\left( \textbf{x} \right) = \prod_{i= 1}^{k} \left(
	1-f_{i}\left( \textbf{x} \right) ^{\left| F \right| -1} \right)$. Then,
	repeatedly applying the substitution $ x_{j}^{\left| F \right| } \to
	x_{j}$ to $g$ yields a polynomial $\overline{g} = g$ for all
	$\textbf{x}\in F^{n}$. Furthermore, $ \deg _{ x_j} \left(  \overline{g}
	\right) \le \left| F \right| -1$ for $1 \le j \le n$(This is clear as ,
	if it were not we would be able to apply the substitution once again).
	Then, substituting $\overline{g}$ in place of $g$ yields \[ M =
	\sum_{\textbf{x } \in F^{n}}^{} \overline{g}\left( \textbf{x} \right)
.\] Then, applying our lemma, we see that all monomials with degree $\left| F
\right|  -2$ or less will equal $0$ and hence the only possible nonzero terms
of $\overline{g}$ are those of the form $ \prod_{i= 1}^{n}  x_{i}^{\left| F
\right| -1}$. Expanding the product, we see that such a monomial would be of
degree $n\left(\left| F \right| - 1 \right)$, however as $\deg f_{i} ^{\left| F
\right| -1} = \left( \left| F \right| -1 \right) \deg \left( f_{i} \right)$, we
see that $\deg \left( g \right) \le \left( \left| F \right| -1 \right) \sum_{i=
1}^{k} f_{i} < n\left( \left| F \right| -1 \right)$ by construction.
Consequently, any such monomial of $\overline{g}$ (and hence $g$) will have a
zero coefficient, and thus $M = N \equiv 0 \left( \text{mod } p \right) $.
\end{proof} This theorem allows us to prove certain constraints (and in the
case of prime  $p$, the exact value) on the size of the davenport constant for
a certain group. This result is too advanced for now, but we will return to it
later. Next, let us examine theorems concerning sumsets such as the
Cauchy-Davenport Theorem and the Erdős-Heilbronn conjecture on the lower bound
of the size of sumsets.\ \begin{theorem}[Cauchy-Davenport Theorem] Given a
	prime $p$ and nonempty $A,B \subseteq Z_{p}$, then $\left| A + B
	\right| \ge \min \left\{ p, \left| A \right|  + \left| B \right|  -1
	\right\} $.  \end{theorem} \begin{proof}[Proof] Suppose $\left| A
	\right| + \left| B \right| > p$, we must have that for any element $x
	\in Z_p$, $\left( A \right) \cap \left( x \setminus B \right) \neq \O$
	(As there are only $p$ possible elements which could be in each set).
	Hence, $A + B = Z_p$. Thus, let us assume $\left| A \right|  + \left| B
	\right| \le p$ and suppose indirectly that $\left| A+B \right| \le
	\left| A \right| + \left| B \right| -2$. Let $A+B \subseteq C \subseteq
	Z_p$ such that $\left| C \right| = \left| A \right|  + \left| B \right|
	-2$. Next, define $f\left( x,y \right)  = \prod_{c \in C}^{} \left( x +
	y - c \right) $ and note that we must have $f\left( a,b \right)  = 0$
	for all $\left( a,b \right)  \in A \times B$ as $A,B \subseteq C$. Now, note
	that $\deg \left( f \right) = \left| C \right| = \left| A \right|  +
	\left| B \right|  -2$, and hence $[x^{\left| A \right| -1}y ^{\left| B
	\right| -1}]f\left( x,y \right) = \binom{\left| A \right| + \left| B
\right|  -2}{\left| A \right| -1} \neq 0$ as $\left| A \right| -1 < \left| A
\right|  + \left| B \right| -2 < p$. Hence, by the Combinatorial
Nullstellensatz (Theorem 1.2), we must have a pair $\left( a,b \right)  \in A
\times B$ such that $f\left( a,b \right) \neq 0$ $\lightning$. Thus, the theorem must be true.
\end{proof} With this basic theorem about
sumsets proven, we now take a look at restricted sumsets, those being sumsets
where a sum is excluded if it satisfies a certain property, normally being the
root of a particular polynomial.  \begin{notation}[Restricted Sumset] For a
	polynomial $h \left( x_0, x_1, \ldots, x_{k} \right) $ and subsets
	$A_0, A_1, \ldots, A_{k} \subseteq Z_p$ define \[ \oplus _h
	\sum_{i=0}^{k} A_{i} = \left\{ a_0 + a_1 + \ldots + a_{k} : a_{i} \in
A_{i}, h\left( a_0, a_1, \ldots, a_{k} \right) \neq 0 \right\} \] to be the
restricted sumset over the $A_{i}$s with respect to $h$.  \end{notation}
\begin{theorem}[General Restricted Sumset Theorem] For a prime $p$ and a
	polynomial $h\left( x_0, .., x_{k} \right)$ over $Z_p$ and nonempty
	$A_0, A_1, \ldots, A_{k} \subseteq Z_p$, define  $c_{i} = \left| A_{i}
	\right| -1$ and $m = \sum_{i=0}^{k} c_{i} - \deg \left( h \right)$.\\
	If $[x_0^{c_0}\ldots x_{k}^{c_{k}}]((\sum_{i=0}^{k} x_{i})^{m} \cdot
	h\left( \textbf{x} \right) ) \neq 0$, then \[ \left|  \oplus _h
		\sum_{i=0}^{k}  A_{i} \right| \ge m+1 \ \text{ (consequently m
		< p)} .\] \end{theorem} \begin{proof}[Proof] Suppose indirectly
		that the inequality does not hold (and hence $m \ge p$), then
		we may define $E \subseteq Z_p$ such that $E$ is a multi-set
		containing $m$ elements and $\oplus_h \sum_{i=0}^{k} A_{i}
		\subseteq E$. Let $Q\left( \textbf{x} \right) = h\left(
			\textbf{x} \right) \prod_{e \in E}^{}  \left(
			\sum_{i=0}^{k} x_{i} - e \right)  $.  By our
			construction we must have that $Q\left( \textbf{x}
			\right)  = 0$ for all $\textbf{x} \in \prod_{i=0}^{k}
			A_{i}$ as either $h\left( \textbf{x} \right)  = 0$ or
			$\sum_{i=0}^{k} x_{i} \in \oplus_h \sum_{i=0}^{k}
			A_{i} \subseteq E$. Furthermore $\deg \left( Q \right)
			= m + \deg \left( h \right) = \sum_{i=0}^{k} c_{i}$ by
			construction. From this we see $m \ge \sum_{i=0}^{k}
			c_{i}$ and hence $[x_0^{c_0}\ldots c_{k}^{c_{k}}] Q
			\neq 0$ (as it is binomial in nature).\\ Therefore,
			applying Combinatorial Nullstellenstaz (Theorem 1.2)
			yields an $ \textbf{a}\in A$ such at $Q\left(
			\textbf{a} \right) \neq 0 $ $\lightning$. Thus $m < p$
			and $\left| \oplus _ h \sum_{i=0}^{k}  A_{i}  \right|
			\ge m+1$.  \end{proof} With this powerful result proven
			let us now take specific functions for $h$ and prove
			superior lower bounds where possible. First, we examine
			the function $h\left( a_0, \ldots, a_{k} \right)  =
			\prod_{0\le i < j \le n}^{} \left( a_{i} - a_{j}
			\right) $: \begin{theorem}[Restricted Sumset Theorem]
				For a prime $p$, nonempty $A_0, A_1, \ldots,
				A_{k} \subseteq Z_p$ with  $\left| A_{i}
				\right| \neq \left| A_{j} \right| $ for any $i
				\neq j$ and for the $h$ defined above. If
				$\sum_{i=0}^{k} \left| A_{i} \right| \le p +
				\binom{k+2}{2} -1$, then \[ \left| \oplus_h
				\sum_{i=0}^{k} A_{i} \right| \ge \sum_{i=0}^{k}
			\left| A_{i} \right| - \binom{k+2}{2} + 1 .\]
		\end{theorem} A special case of this theorem for only two sets
		is the following: \begin{theorem}[Erdős-Heilbronn Conjecture]
			For a prime $p$ and nonempty $A,B \subseteq Z_{p}$,
			then $\left| A \oplus_{h} B \right| \ge \min \left\{ p,
			\left| A \right|  + \left| B \right|  - \delta \right\}
			$ where $\delta = 3$ for  the case $A = B$ and $\delta
			= 2$ in all other cases.  \end{theorem} In order to
			prove these theorems, let us first state and prove a
			lemma concerning the coefficient of a particular
			polynomial: \begin{lemma}[] Let $0 \le c_0, \ldots,
				c_{k} \in \Z$ and define $m = \sum_{i=0}^{k}
				c_{i} - \binom{k+1}{2}$ (it is trivial that $m$
				is nonnegative). Then,\[[x_0^{c_0}\ldots
				x_{k}^{c_{k}}] \left( \left( \sum_{i=0}^{k}
			x_{i} \right) ^{m} \prod_{k \ge i > j \ge 0}^{} \left(
x_{i} - x_{j} \right)  \right) = \frac{m!}{c_0! c_1! \ldots c_{k}!} \prod_{k
\ge i > j \ge 0}^{} \left( c_{i} - c_{j} \right).\] \end{lemma} \todo{Provide
proof of this lemma based on proof of Ballot problem} With this out of the way,
we now prove proposition 2.4: \begin{proof}[Proof of Restricted Sumset Theorem]
	For this proof we will take the aforementioned \[ h\left( \textbf{x}
		\right) = \prod_{0\le i < j \le k}^{} \left( x_{i} - x_{j}
	\right) .\] Now, let us define $c_{i} = \left| A_{i} \right|  -1$ and
	$m = \sum_{i=0}^{k} c_{i} - \binom{k+1}{2}$. Rearranging the
	assumptions of this theorem yields $\sum_{i=0}^{k} \left| A_{i} \right|
	- \binom{k+2}{2} + 1 \le p$ and, applying the trivial combinatorial
	identity $\binom{k+2}{2} =  \binom{k+1}{2} +  (k+1)$ yields \\
	$\sum_{i=0}^{k} c_{i} - \binom{k+1}{2} + 1 = m + 1 \le p$ (hence $m <
	p$). Then  \[ [x_0^{c_0}\ldots x_{k}^{c_{k}}]\left( \left(
	\sum_{i=0}^{k} x_{i} \right) h \right) =  \frac{m!}{c_0! \ldots
	c_{k}!}\prod_{0 \le i < j \le k}^{} \left( c_{i} - c_{j} \right) .\] We
	know this product to be nonzero modulo  $p$ as  $c_{i} \neq c_{j}$  for
	$i \neq j$ by construction and $m < p$. Finally, as the coefficient is
	nonzero and as $\deg \left( h \right) =  \binom{k+2}{2}$ (as there are
	$k+2$ possible $x_{i}$'s and each term of the product will contain two
	distinct $x_{i}$'s so there are $\binom{k+2}{2}$ terms each of degree
	$1$), we have $m = \sum_{i=0}^{k} c_{i} - \deg \left( h \right)$
	\todo{Ask grynkiewicz about this as Alons paper says $+ \deg \left( h
	\right)$ instead of minus}. Hence, applying theorem 2.3 yields  $\left|
	\oplus _h \sum_{i= 0}^{k}  A_i \right| \ge m + 1 = \sum_{i=0}^{k}
	\left| A_{i} \right| - \binom{k+2}{2}+1$ by construction.  \end{proof}
We conclude this section with the proof of the Erdős-Heilbronn Conjecture:
\begin{proof}[Proof of Erdős-Heilbronn Conjecture]
	\todo{Add proof}
\end{proof}
