%%%%%%%%%%%%%%%%%%%%%%%%%%%%%%%%%%%
%          Introduction           %
%%%%%%%%%%%%%%%%%%%%%%%%%%%%%%%%%%%
\chapter{Introduction and Preliminary Results}\label{chap:intro}
\pagenumbering{arabic}
\setcounter{section}{0}
\setcounter{subsection}{0}
\setcounter{page}{1}

\section{Introduction}

One of the earliest objects of algebraic study, real (specifically integral) univariate polynomials generalize the most simple algebraic objects, linear functions, and have significant application. Beyond this, multivariate polynomials offer further generalization. Expanding even further, multivariate polynomials over arbitrary rings (and fields) prove surprisingly useful. In this thesis, we will showcase a variety of methods and applications involving polynomials. These applications will prove some surprising results, both abstract and concrete. To understand these proofs, one need only thorough knowledge of algebraic tools and structures, and some experience with methods of combinatorial proof.

Merging the fields of Combinatorics and Number Theory yields Combinatorial Number Theory and Additive Combinatorics, two overlapping fields answering questions pertaining to zero-sums over sequences and zeroes of arbitrary polynomials as well as other problems at the intersection of Combinatorics, Algebra, and Number Theory. We begin Chapter 1 by presenting many common objects which we will make use of throughout the thesis. This will include some objects of common study, such as polynomials, as well as some notation and conventions common to the field, and the statements of a few common theorems which the reader will likely be familiar with. After this, in chapter 2, we introduce the Nullstellensatz, literally translating as zero-locus theorems, a series of theorems concerning the number of common zeroes of a family of polynomials. These theorems, while extremely powerful, will have relatively simple statements and proofs culminating in the Punctured Nullstellensatz.
Moving on, in chapter 3 we demonstrate some applications of these Nullstellensatz. These will mostly make use of the simplest version of the Nullstellensatz which we present, Alon's Combinatorial Nullstellensatz II \cite{alon_1999}. The applications will come in two main varieties, sumset proofs and graph theoretic proofs. This chapter will provide some of the most concrete applications of the polynomial method. In chapter 4 we present Dyson's theorem \cite{dyson_1963} on laurent polynomials and Zeilberger and Bressoud's q-Dyson variant, with a proof based on a concrete version of the Combinatorial Nullstellensatz \cite{qdyson_2014}. In chapter \(5\) we provide Reiher's proof of the Kemnitz conjecture \cite{reiher_2007}, on the minimal size of a set of ordered pairs required to have a subset of size \(p\) whose elements are zero-sum over the finite field \(F_{p}\). This proof will make heavy use of the theorem of Chevalley and Warning, our other major trick in the bag of polynomial methods concerning zero-sum methods. In chapters \(6\) and \(7\) we prove some longer results from additive combinatorics concerning zero-sum sequences, and in chapter \(8\) we provide a few nontrivial applications of these methods in other areas. Overall, we aim to provide a wide breadth of results demonstrating the power of these polynomial methods in a diverse range of fields. First, though, in chapter \(2\) we will provide proofs of these results which will prove so powerful in later sections.
\newpage
\section{Definitions, Notation, and Preliminary Results}
Before we continue on, we define a few basic objects and results in use throughout the thesis. Readers are assumed to have knowledge equivalent to a first course in group and ring theory and a first course in combinatorics.
\begin{definition}[Polynomial Ring]
	Given a ring \(R\), we define \(R\left[ x \right] \) to be the ring generated by all formal sums of the form \(f = \sum_{i=0}^{\infty} c_{i} x^{i}\), \( c_{i} \in R\) where there is a \(N \in \N_{0}\) so that \(c_{i} = 0\) for all \(i \ge N\) with ring operations given by the standard componentwise formulae. We call \(R\left[ x \right] \) \textbf{the polynomial ring} in \(1\) variable over \(R\).
\end{definition}
\begin{definition}[Multivariate Polynomial Ring]
Given a ring \(R\), we define a \textbf{polynomial ring} over \(n\) variables, \(R\left[ x_1, \ldots, x_{n} \right] \) inductively by the rules \( R\left[ x_1 \right] = R\left[ x \right] \) and \(R\left[ x_1, \ldots, x_{i} \right] = \left( R\left[ x_1, \ldots, x_{i-1} \right]  \right) \left[ x_{i} \right]   \).
\end{definition}
\begin{definition}[Laurent Polynomial]
	Given a field \(F\), we define a \textbf{laurent polynomial ring} in \(n\) variables over \(F\) as \(F\left[ x_1, \ldots, x_{n}, x_1^{-1}, \ldots, x_{n}^{-1} \right] \).
\end{definition}
\begin{definition}[Monomial]
Given a ring \(R\), a polynomial \(f \in R\left[ x_1, \ldots, x_{n} \right] \)	is a monomial if \(f = \prod_{i \in J }^{} x_{i}\) where \(J \subseteq \{1, \ldots, n\} \).
\end{definition}
\begin{definition}[Degree]
	We define the \textbf{degree} of a polynomial \(f \in R\left[ x \right] \) as \[\deg \left( f \right) = \min \{N \in \N_{0}: c_{i} = 0 \ \forall \ i \ge N\} .\] By convention the zero polynomial has degree \(-1\).
	For multivariate polynomial rings there are two natural generalizations of degree, total degree and projected degree. For a polynomial \[f  = \sum_{i_1, \ldots, i_{n} \in N_0}^{} c_{i_1, \ldots, i_{n}} x_1^{i_1} \ldots x_{n}^{i_{n}} \in R\left[ x_1, \ldots, x_{n} \right] \] we define the \textbf{total degree} as \[\deg  \left(  f \right)  = \min \{N \in N_0 : c_{i_1, \ldots, i_n} = 0 \ \forall \ i_1, \ldots, i_{n} \in N_{0} \text{ so that } \sum_{j=1}^{n} i_{j} \le N  \} \] and we define the \textbf{projected degree in variable \(x_{k}\)} by letting \[J = \left( R\left[ x_1, \ldots, x_{k-1}, x_{k+1}, \ldots, x_{n} \right]  \right) \left[ x_{k} \right] \] which one can readily verify is the same as \(R\left[ x_1, \ldots, x_{n} \right] \). Then, \[
	\deg _{x_{k}} \left( f \right) = \deg_{J} \left(  f \right)
\] where \(\deg _{J}\left( f  \right) \) is simply the degree over the given ring \(J\).
\end{definition}
\begin{definition}[Module]
	Given a ring \(R\), we define a (left) \(R\)-module \(M\) to be a set endowed with two operations, addition and multiplication, such that \(+: M \times M \to M\) and \(\cdot: R \times M \to M\) with \(+\) satisfying the axioms of an abelian group and \(\cdot\) being associative, having an identity, and having both left and right distributive laws (\(\left( r+s \right) x = rx + sx\) and \(r\left( x+y \right) = rx + ry\) for \(r, s \in R\) and \(x, y \in M\)). If \(R\) is a module, we instead call \(M\) a vector space.\\
	In a vector space \(M\), we call the set \(x_1, \ldots, x_{n}\) a basis of \(M\) if is linearly independent and spans \(M\).
\end{definition}
\begin{definition}[Zero-Sum]
	Given \(n, p \in \N\) and a subset \(I \subseteq \Z^{n}\), we say a \(I\) is zero-sum if \(\sum_{}^{} I = \sum_{i  \in I}^{} i \equiv 0 \mod  p  \).
\end{definition}
\begin{definition}[Characteristic Function]
	Given a vector \(\textbf{x}\), the characteristic function \(\chi_{\textbf{x}}\)	has \(\chi_{\textbf{x}}\left( \textbf{y} \right)  = \left \{
		\begin{array}{11}
			1, & \quad  \textbf{x}= \textbf{y}\\
			0, & \quad \textbf{x} \neq \textbf{y}
		\end{array}
		\right.\)
\end{definition}
\begin{definition}[Map Functor]
		For a field \(F\) and \(m\ge 0\), define \(\MAP\left( \{0, 1\} ^{m}, F \right) = \{f: \{0, 1\} ^{m} \to F | f \text{ is a function}\}  \).
\end{definition}
\begin{notation}[Polynomial Coefficient]
	For a laurent polynomial \[f = \sum_{i_1, \ldots, i_{n} \in \Z}^{} c_{i_1, \ldots, i_{n}x_1^{i_1} \ldots x_{n}^{i_{n}}\] over \(n\) variables, we denote the coefficient of a particular term by \(\left[ x_1^{i_1}\ldots x_{n}^{i_{n}} \right] f \coloneqq c_{i_1, \ldots, i_{n}}\) when it is more convenient.
\end{notation}
\begin{notation}[Vector Polynomials]
	For a laurent polynomial \(f\left( x_1, \ldots, x_{n} \right) \) evaluated at \(x_1, \ldots, x_{n}\), we implicitly denote \(\textbf{x} = \left( x_1, \ldots, x_{n} \right) \) and \(f\left( \textbf{x} \right)  = f\left( x_1, \ldots, x_{n} \right) \) when it is unambiguous and convenient.
\end{notation}
\begin{notation}[Congruence]
	For a congruence statement \(a \equiv b \left( \mod c \right) \) we sometimes shorten it to \(a \equiv b\) when the modulus \(c\) is unambiguous.
\end{notation}
\begin{notation}[Integer subsets]
	Given an upper bound \(p\), we denote the set of all natural numbers less than or equal to \(p\) by \(\left[ 1, p \right] = \{x \in \Z: 1 \le x \le p\} \). In general, \(\left[ q, p \right]  = \{x \in \Z : q \le x \le p\} \). This notation will only be used when it is unambiguous with the normal notation for closed intervals.
\end{notation}
\begin{notation}[Sum over set]
	For a set \(A\) on which a sum is well defined, we define \(\sum_{}^{} A = \sum_{a \in A}^{} a\) where it is unambiguous. Similarly, \(\sum_{A}^{} f\left( a \right)  = \sum_{a \in A}^{} f\left( a \right) \).
\end{notation}
The next theorem has many equivalent statements as well as a few stronger (and weaker) ones. For the rest of this thesis, we will assume the Fundamental Theorem of Algebra refers to the following statement.
\begin{theorem}[Fundamental Theorem of Algebra]
	Given a ring \(R\) and \(f \in R\left[ x \right] \)	with \(\deg \left( f \right)  = n \ge 0\), let \(X = \{x \in R : f\left( x \right)  = 0\} \). Then, \(\left| X \right| \le n\).
\end{theorem}
